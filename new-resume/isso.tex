
  \documentclass[10pt,a4paper,ragged2e]{altacv}

  % Change the page layout if you need to
  \geometry{left=1cm,right=9cm,marginparwidth=6.8cm,marginparsep=1.2cm,top=1cm,bottom=1cm}

  % Change the font if you want to, depending on whether
  % you're using pdflatex or xelatex/lualatex
  \ifxetexorluatex
    % If using xelatex or lualatex:
    \setmainfont{Carlito}
  \else
    % If using pdflatex:
    \usepackage[utf8]{inputenc}
    \usepackage[T1]{fontenc}
    \usepackage[default]{lato}
  \fi

  % Change the colours if you want to
  \definecolor{VividPurple}{HTML}{3E0097}
  \definecolor{SlateGrey}{HTML}{2E2E2E}
  \definecolor{LightGrey}{HTML}{666666}
  \colorlet{heading}{VividPurple}
  \colorlet{accent}{VividPurple}
  \colorlet{emphasis}{SlateGrey}
  \colorlet{body}{LightGrey}

  % Change the bullets for itemize and rating marker
  % for \cvskill if you want to
  \renewcommand{\itemmarker}{{\small\textbullet}}
  \renewcommand{\ratingmarker}{\faCircle}
  \newcommand{\SubItem}[1]{
    {\setlength\itemindent{15pt}\item[-] #1}
}

  \addbibresource{sample.bib}

  \pdfmapfile{+fontawesome5.map}

  \begin{document}

  \name{\huge Mehtab Zafar}
  \tagline{Product Security Engineer}
  \personalinfo{%
    \hspace{0.6cm}\email{\normalsize mehtabzafar@mzfr.me}
    \hspace{1.1cm}\linkedin{\normalsize\href{https://www.linkedin.com/in/mzfr/}{linkedin.com/in/mzfr/}}

    \hspace{1.3cm} \hspace{1.5cm}
  }

  \begin{fullwidth}
  \makecvheader
  \end{fullwidth}

  \AtBeginEnvironment{itemize}{\small}

  \cvsection[page1sidebar]{Professional Experience}
  \cvevent{Product Security Engineer}{Red Queen Dynamics}{July 2021 -- Aug 2023}{Washington, D.C}
  \begin{itemize}
      \item Worked as a full-stack dev to develop from scratch a security training product (using Django/PostgreSQL) and created a CI/CD pipeline for automated deployments.
      \item Handled DevOps for the team i.e. created and managed cloud infrastructure (AWS) to support internal applications.
      \item Performed ad-hoc penetration tests for various clients.
  \end{itemize}
  \divider
  \cvevent{BugBounty}{Hackerone / Bugcrowd / Intigriti}{Aug 2020 -- ongoing}{Independent/Remote}
  \begin{itemize}
  
  \item Listed on Hall of Fame for various companies, such as \textbf{Google}, GitHub, PayPal, US Department of Defense, DELL, Atlassian, Zynga.
\item Performed static code analysis to identify various vulnerabilities in APK files.
  \item Performed zero-day research on open-source software.
 
  \end{itemize}

  \divider

  \cvevent{Developer}{The Honeynet Project {\hspace{0.5cm}\color{accent} \faGithub \href{https://github.com/mushorg/tanner/commits?author=mzfr}{\small \normalfont \hspace{1mm}Code}}}{May 2020 -- August 2020}{Google Summer of Code}
  \begin{itemize}
  \item Improved the speed and functionality of a high-interaction honeypot (Snare/Tanner)

  \item Added support for persistent storage using PostgreSQL and SQLAlchemy
  
  \item Improved the API functionality based on the new database structure
  
  \item Both the API and honeypot were written in Python and used libraries like asyncio, Redis, jinja, etc
  \end{itemize}


  \divider

  \cvevent{Devops}{Vulnhub / TryHackMe}{Aug 2019 -- March 2020}{Independent/Remote}
  \begin{itemize}
    
    \item Created various CaptureTheFlag (CTF) challenges for \href{https://tryhackme.com/p/falconfeast}{TryHackme.com} that teach about Web-related vulnerabilities like XXE, XSS, JWT.

    \item Created potential Vulnerable machines for \href{https://www.vulnhub.com/series/djinn,260/}\emp{VulnHub.com}
    
    \item All the virtual machines had custom applications made in Python \& bash
  
    \item Invited to perform beta test various vulnerable virtual machines like TempusFugit series, DC8 for vulnhub.com.

  \end{itemize}

  \divider

  \cvevent{Developer}{XBMC Foundation {\hspace{0.5cm}\color{accent} \faGithub \href{https://github.com/xbmc/addon-check/commits?author=mzfr}{\small \normalfont \hspace{1mm}Code}}}{May 2018 -- Aug 2018}{Google Summer of Code}
  \begin{itemize}

  \item Worked on an Open-source project under a student program by Google.

  \item Developed a tool for performing static code analysis on all addons for Kodi, written in Python.

  \end{itemize}


  \cvsection{Projects}

  \begin{itemize}
  \item {\color{black} \normalsize \textbf{slicer}} \hfill {\color{accent} \faGithub \href{https://github.com/mzfr/slicer}{ \hspace{0.5mm}Code}}

  \begin{itemize}
      \item Wrote this tool to automate the bug-hunting process on Android applications (APK).
      \item It can find possible vulnerable activities, receivers, and services.
  \end{itemize}

  \divider

  \item {\color{black} \normalsize \textbf{liffy}} \hfill {\color{accent} \faGithub \href{https://github.com/mzfr/liffy}{ \hspace{0.5mm}Code}}

    \begin{itemize}
        \item Wrote this tool to automate the process of discovering and exploiting\\ Local file inclusion (LFI) attack which can then be leveraged to get a reverse shell.
\end{itemize}

  \divider

%   \item {\color{black} \normalsize \textbf{takeover}} \hfill {\color{accent} \faGithub \href{https://github.com/mzfr/takeover}{ \hspace{0.5mm}Code}}
%     \begin{itemize}
%         \item Made a CLI tool to check subdomain takeover at a mass scale.
%         \item Given output of amass/subfinder this tool can check if any of the subdomain is vulnerable to subdomain takeover.
%     \end{itemize}

%   \end{itemize}

  \clearpage

  \end{document}
